\chapter{Introdução}\label{chp:LABEL_CHP_1}

\section{Motivação}\label{sec:LABEL_CHP_1_SEC_A}

Nos dias de hoje, em que a utilização da internet está cada vez maior, nos deparamos com usuários de diferentes níveis estão tomando parte no mundo digital. Neste cenário, temos cada vez mais sistemas web disponíveis na rede e interesses individuais naquele produto permanecer no ar. Mas será que esse interesse não deveria vir acompanhado com uma necessidade em manter essas aplicações testadas? Quando um cliente solicita para uma empresa que esta desenvolva o seu projeto, depois dele pronto, após alguns meses como o cliente saberá que não existem páginas retornando 404? Páginas com erro? Ou até mesmo o sistema não está disponível?! Foram a partir dessas necessidades que surgiu a motivação do projeto. 

Temos além disso desenvolvedores responsáveis pela manutenção de alguns desses códigos "legados", que não foram eles que desenvolveram mas que está na mão deles para a sua manutenção. Para identificar que esse cenário realmente ocorria na vida real realizamos uma análise dentro do mercado de desenvolvimento no Brasil. Nos utilizamos de fóruns e grupos de desenvolvedores no Facebook para aplicar pesquisas e assim determinar se essa análise era correta e identificava uma fatia real dos desenvolvedores brasileiros.

