No contexto de desenvolvimento de aplicações, muitas vezes nós, como desenvolvedores destas e/ou responsáveis pela sua manutenção, nos deparamos com situações onde existe muito interesse em desenvolver testes automatizados para o projeto, mas há muitos empecilhos que no dia-a-dia nos impedem de aplicar essa prática no nosso cotidiano. 

Dado em vista que para começar a desenvolver testes automatizados para a sua aplicação a curva de aprendizado é alta, realizamos um estudo de caso com diversas comunidades de desenvolvedores para que pudéssemos identificar quais são as dificuldades em adotar essa prática, e como podíamos encontrar uma forma de auxiliá-los a incorporar a prática de testes no seu ambiente de trabalho trazendo um alto ROI(Return Over Investiment) entregando testes de valor de forma rápida e fácil.

O objetivo desse trabalho foi criar uma aplicação onde seria possível realizar testes da aplicação já em produção, na nuvem. De forma rápida e sem precisar codificar.